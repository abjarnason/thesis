\chapter{Environment}
\label{chapter:environment}

A problem instance is rarely totally independent of its environment.
Most often you need to describe the environment you work in, what
limits there are and so on. This is a good place to do that. First we
tell you about the LaTeX working environments and then is an example
from an thesis written some years ago.


\section{LaTeX working environments}
\label{sec:environments}

To create \LaTeX\ documents you need two things: a \LaTeX\ environment for
compiling your documents and a text editor for writing them.

\subsection{Environment}

Fortunately \LaTeX\ can nowadays be found for any (modern) computer
environment, be it Linux, Windows, or Macintosh.
For Linuxes (and other Unix clones) and Macs, I'd recommend \emph{TeX
Live}~\cite{TeXLive}, which is the current default \LaTeX\ distribution for
many Linux flavors such as Fedora, Debian, Ubuntu, and Gentoo.
TeX Live is the replacement for the older \emph{teTeX}, which is
no longer developed.

TeX Live works also for Windows machines (at least according to their web
site); however, I have used \emph{MiKTeX}~\cite{MiKTeX} and can recommend it
for Windows. 
MiKTeX has a nice package manager and automatically fetches missing packages
for you.

\subsection{Editor}

You can write \LaTeX\ documents with any text editor you like, but having
syntax coloring options and such really helps a lot.
My personal favourite for editing \LaTeX\ is the
\emph{TeXlipse}~\cite{TeXlipse} plugin for the Eclipse IDE~\cite{Eclipse}. 
Eclipse is an open-source integrated development environment (IDE) initially
created for writing Java code, but it currently has support for editing
languages such as C, C++, JavaScript, XML, HTML, and many more. 
The TeXlipse plugin allows you to edit and compile \LaTeX\ documents directly
in Eclipse, and compilation errors and warnings are shown in the Eclipse
\emph{Problems} dialog so that you can locate and fix the issues easily.
The plugin also supports reference traversal so that you can locate the source
line where a label or a citation is defined.

Eclipse is an entire development environment, so it may feel a bit heavy-weight
for editing a document. 
If you are looking for a more light-weight option, check out TeXworks. 
TeXworks is a \LaTeX\ editor that is packaged with the newer MiKTeX
distributions, and it can be acquired from \url{http://www.tug.org/texworks/}.

And if you are attached to your \emph{emacs} or \emph{vim} editor, you
can of course edit your \LaTeX\ documents with them. 
Emacs at least has syntax coloring and you can compile your document with a key
binding, so this may be a good option if you prefer working with the standard
Linux text editors.

\section{Graphics}

When you use \texttt{pdflatex} to render your thesis, you can include PDF images
directly, as shown by Figure~\ref{fig:indica_model} below.

\begin{figure}[ht]
  \begin{center}
    \includegraphics[width=\textwidth]{images/indica_model.pdf}
    \caption{The INDICA two-layered value chain model.}
    \label{fig:indica_model}
  \end{center}
\end{figure}

You can also include JPEG or PNG files, as shown by Figure~\ref{fig:eeyore}.

\begin{figure}[ht]
  \begin{center}
    \includegraphics[width=9cm]{images/ihaa.jpg}
    \caption{Eeyore, or Ihaa, a very sad donkey.}
    \label{fig:eeyore}
  \end{center}
\end{figure}

You can create PDF files out of practically anything. 
In Windows, you can download PrimoPDF or CutePDF (or some such) and install a
printing driver so that you can print directly to PDF files from any
application. There are also tools that allow you to upload documents in common
file formats and convert them to the PDF format.
If you have PS or EPS files, you can use the tools \texttt{ps2pdf} or
\texttt{epspdf} to convert your PS and EPS files to PDF\@.

% Comment: If your sentence ends in a capital letter, like here, you should
% write \@ before the period; otherwise LaTeX will assume that this is not
% really an end of the sentence and will not put a large enough space after the
% period. That is, LaTeX assumes that you are (for example), enumerating using
% capital roman numerals, like I. do something, II. do something else. In this
% case, the periods do not end the sentence.

% Similarly, if you do need a normal space after a period (instead of
% the longer sentence separator), use \  (backslash and space) after the
% period. Like so: a.\ first item, b.\ second item.

Furthermore, most newer editor programs allow you to save directly to the PDF
format. For vector editing, you could try Inkscape, which is a new open source
WYSIWYG vector editor that allows you to save directly to PDF\@. 
For graphs, either export/print your graphs from OpenOffice Calc/Microsoft
Excel to PDF format, and then add them; or use \texttt{gnuplot}, which can
create PDF files directly (at least the new versions can).
The terminal type is \emph{pdf}, so the first line of your plot file should be
something like \texttt{set term pdf \ldots}.

To get the most professional-looking graphics, you can encode them using the
TikZ package (TikZ is a frontend for the PGF graphics formatting system).
You can create practically any kind of technical images with TikZ, but it has a
rather steep learning curve. Locate the manual (\texttt{pgfmanual.pdf}) from
your \LaTeX\ distribution and check it out. An example of TikZ-generated
graphics is shown in Figure~\ref{fig:page-merge}.

\begin{figure}[ht]
  \begin{center}
    \newcommand*{\actver}{\smash{\ensuremath{v_{\text{\textit{active}}}}}}

\tikzstyle{lbl}=[font=\scriptsize,midway,sloped]
\tikzstyle{opendash}=[densely dotted,thick] 
\tikzstyle{opendeco}=[decoration={zigzag,amplitude=0.1em,segment length=0.6em}]
\tikzstyle{actverline}=[dashed]
\tikzstyle{entryline}=[densely dotted]
\tikzstyle{area}=[ellipse,draw,dashed]
\tikzstyle{liveentry}=[entryline,postaction={%
  decorate,%
  decoration={%
    markings,%
    mark=at position .5pt with{\arrowreversed[line width=.35pt]{|}};,
    mark=at position 1 with{%
      \arrow[line width=.8pt]{stealth}};%
  }%
}]
\tikzstyle{deadentry}=[entryline,postaction={%
  decorate,%
  decoration={%
    markings,%
    mark=at position .5pt with{\arrowreversed[line width=.35pt]{|}};,
    mark=at position 1 with{%
      \arrow[line width=.8pt]{stealth}};%
%      \arrow[line width=.35pt,black,fill=white]{*}};% 
  }%
}]
\tikzstyle{pageborder}=[thick]
\tikzstyle{pagename}=[anchor=center,text=black!50,font=\huge]
\newlength{\spexx}
\setlength{\spexx}{1.2cm}
\newlength{\spexy}
\setlength{\spexy}{.5cm}

\subfigure[Before]{
\begin{tikzpicture}[x=\spexx,y=\spexy,%
  every pin edge/.style={draw,dotted},%
  pin distance=.5\spexx]
  \small
  \coordinate (lo) at (0,-5);
  \coordinate (o) at (0,0);
  \coordinate (hi) at (0,7);
  \coordinate (loact) at (3,-5);
  \coordinate (oact) at (3,0);
  \coordinate (hiact) at (3,7);
  \coordinate (loinf) at (4,-5);
  \coordinate (oinf) at (4,0);
  \coordinate (hiinf) at (4,7);
  
  \draw[->] (lo) -- node[below,lbl] {Versions} ($(loinf) + (1em,0)$);
  \draw[->] (lo) -- node[above,lbl] {Keys} ($(hi) + (0,1em)$);

  \draw[pageborder] (lo) -- (o) -- (hi);
  \draw[pageborder] (lo) -- (loinf);
  \draw[pageborder] (hi) -- (hiinf);
  \draw[pageborder] (o) -- (oinf);

  \draw[opendash] decorate [opendeco] { (loinf) -- (oinf) };
  \draw[opendash] decorate [opendeco] { (oinf) -- (hiinf) };

  \node[pagename] at (2,3.5) {$p$};
  \node[pagename] at (2,-2.5) {$s$};

  \draw[actverline] ($(loact) + (0,-1em)$) node[below=.4em] {\actver} --
    ($(hiact) + (0,1em)$);

  % Page p contents
  \draw[deadentry] (0,6) -- (1,6);
  \draw[deadentry] (1,6) -- (2,6);
  \draw[deadentry] (2,6) -- (3,6);
  \draw[liveentry] (0,5) -- (4,5);
  \draw[deadentry] (0,4) -- (2,4);
  \draw[deadentry] (1,3) -- (3,3);
  \draw[deadentry] (0,2) -- (1,2);
  \draw[deadentry] (2,2) -- (3,2);
  \draw[deadentry] (0,1) -- (2,1);
  \draw[deadentry] (2,1) -- (3,1);

  % Page s contents
  \draw[liveentry] (0,-1) -- (4,-1);
  \draw[deadentry] (0,-2) -- (2,-2);
  \draw[deadentry] (0,-3) -- (3,-3);
  \draw[liveentry] (3,-4) -- (4,-4);

  \node[area,minimum width=4.5\spexx,minimum height=.6\spexy,pin=178:{$1$}]
    at (2,5) {};
  \node[area,minimum width=4.5\spexx,minimum height=.6\spexy,pin=182:{$2$}]
    at (2,-1) {};
  \node[area,minimum width=1.5\spexx,minimum height=.6\spexy,pin=330:{$3$}]
    at (3.5,-4) {};

\end{tikzpicture}}
\subfigure[After]{
\begin{tikzpicture}[x=\spexx,y=\spexy,%
  every pin edge/.style={draw,dotted},%
  pin distance=.5\spexx]
  \small
  \coordinate (lo) at (0,-5);
  \coordinate (o) at (0,0);
  \coordinate (hi) at (0,7);
  \coordinate (loact) at (3,-5);
  \coordinate (oact) at (3,0);
  \coordinate (hiact) at (3,7);
  \coordinate (loinf) at (4,-5);
  \coordinate (oinf) at (4,0);
  \coordinate (hiinf) at (4,7);

  \draw[->] (lo) -- node[lbl,below] {Versions} ($(loinf) + (1em,0)$);
  \draw[->] (lo) -- node[lbl,above] {Keys} ($(hi) + (0,1em)$);

  \draw[pageborder] (lo) -- (o) -- (hi);
  \draw[pageborder] (lo) -- (loinf);
  \draw[pageborder] (hi) -- (hiinf);
  \draw[pageborder] (o) -- (oact);

  \draw[opendash] decorate [opendeco] { (loinf) -- (oinf) };
  \draw[opendash] decorate [opendeco] { (oinf) -- (hiinf) };

  \node[pagename] at (1.5,3.5) {$p$};
  \node[pagename] at (1.5,-2.5) {$s$};
  \node[pagename] at (3.5,1) {$p'$};

  \draw[actverline] ($(loact) + (0,-1em)$) node[below=.4em] {\actver} -- (loact);
  \draw[pageborder] (loact) -- (hiact);
  \draw[actverline] (hiact) -- ($(hiact) + (0,1em)$);

  % Page p contents
  \draw[deadentry] (0,6) -- (1,6);
  \draw[deadentry] (1,6) -- (2,6);
  \draw[deadentry] (2,6) -- (3,6);
  \draw[deadentry] (0,5) -- (3,5);
  \draw[deadentry] (0,4) -- (2,4);
  \draw[deadentry] (1,3) -- (3,3);
  \draw[deadentry] (0,2) -- (1,2);
  \draw[deadentry] (2,2) -- (3,2);
  \draw[deadentry] (0,1) -- (2,1);
  \draw[deadentry] (2,1) -- (3,1);

  % Page s contents
  \draw[deadentry] (0,-1) -- (3,-1);
  \draw[deadentry] (0,-2) -- (2,-2);
  \draw[deadentry] (0,-3) -- (3,-3);

  % Page p' contents
  \draw[liveentry] (3,5) -- (4,5);
  \draw[liveentry] (3,-1) -- (4,-1);
  \draw[liveentry] (3,-4) -- (4,-4);

  \node[area,minimum width=1.5\spexx,minimum height=.6\spexy,pin=10:{$1$}]
    at (3.5,5) {};
  \node[area,minimum width=1.5\spexx,minimum height=.6\spexy,pin=3:{$2$}]
    at (3.5,-1) {};
  \node[area,minimum width=1.5\spexx,minimum height=.6\spexy,pin=330:{$3$}]
    at (3.5,-4) {};
\end{tikzpicture}}

    \caption{Example of a multiversion database page merge. This figure has
    been taken from the PhD thesis of Haapasalo~\cite{HaapasaloThesis}.}
    \label{fig:page-merge}
  \end{center}
\end{figure}

Another example of graphics created with TikZ is shown in
Figure~\ref{fig:tikz-examples}. 
These show how graphs can be drawn and labeled. 
You can consult the example images and the PGF manual for more examples of what
kinds figures you can draw with TikZ. 

% These definitions are only used in the example images; you will not 
% need them for your thesis...
\newlength{\graphdotsize}
\setlength{\graphdotsize}{1.7pt}
\newlength{\graphgridsize}
\setlength{\graphgridsize}{1.2em}
\begin{figure}[ht]
\begin{center}
\subfigure[Examples of obstruction graphs for the Ferry Problem]{
  \newlength{\oggs}
\setlength{\oggs}{1.2\graphgridsize}
\begin{tikzpicture}[x=\oggs,y=\oggs,every pin edge/.style={draw,dotted},pin distance=0.5\oggs,area/.style={ellipse,draw,dashed}] 

% The graph (0,0,5,0)
% o = origo
\coordinate (o) at (0,0);
\coordinate[left=1 of o,pin=100:$q_1$] (q1);
\coordinate[right=1 of o,pin=80:$q_2$] (q2);
\fill[black] (q1) circle (\graphdotsize);
\fill[black] (q2) circle (\graphdotsize);
\foreach \d in {-2, -1, ..., 2} {
  \coordinate (tmp) at ($(o) + (0,\d)$);
  \draw (q1) -- (tmp);
  \draw (q2) -- (tmp);
  \fill[black] (tmp) circle (\graphdotsize);
}
\node[area,minimum height=5.3\oggs,minimum width=0.8\oggs,pin=94:$X_3$] at (o) {};  

% The graph (1,0,3,0)
\coordinate[right=5 of o] (o);
\coordinate[left=1 of o,pin=260:$q_1$] (q1);
\coordinate[left=2 of o] (q1v);
\coordinate[right=1 of o,pin=280:$q_2$] (q2);
\fill[black] (q1) circle (\graphdotsize);
\fill[black] (q2) circle (\graphdotsize);
\fill[black] (q1v) circle (\graphdotsize);
\draw (q1) -- (q1v);
\foreach \d in {-1, 0, 1} {
  \coordinate (tmp) at ($(o) + (0,\d)$);
  \draw (q1) -- (tmp);
  \draw (q2) -- (tmp);
  \fill[black] (tmp) circle (\graphdotsize);
}
\node[area,minimum height=3.3\oggs,minimum width=0.8\oggs,pin=266:$X_3$] at (o) {};  
\node[area,minimum height=0.8\oggs,minimum width=0.8\oggs,pin=260:$X_1$] at (q1v) {};  


% The graph (0,1,3,0)
\coordinate[right=4 of o] (o);
\coordinate[left=1 of o,pin=100:$q_1$] (q1);
\coordinate[right=1 of o,pin=80:$q_2$] (q2);
\coordinate[right=2 of o] (q2v);
\fill[black] (q1) circle (\graphdotsize);
\fill[black] (q2) circle (\graphdotsize);
\fill[black] (q2v) circle (\graphdotsize);
\draw (q2) -- (q2v);
\foreach \d in {-1, 0, 1} {
  \coordinate (tmp) at ($(o) + (0,\d)$);
  \draw (q1) -- (tmp);
  \draw (q2) -- (tmp);
  \fill[black] (tmp) circle (\graphdotsize);
}
\node[area,minimum height=3.3\oggs,minimum width=0.8\oggs,pin=94:$X_3$] at (o) {};  
\node[area,minimum height=0.8\oggs,minimum width=0.8\oggs,pin=80:$X_2$] at (q2v) {};  

% The graph (0,0,3,1)
\coordinate[right=5 of o] (o);
\coordinate[left=1 of o,pin=260:$q_1$] (q1);
\coordinate[right=1 of o,pin=290:$q_2$] (q2);
\fill[black] (q1) circle (\graphdotsize);
\fill[black] (q2) circle (\graphdotsize);
\draw (q1) -- (q2);
\foreach \d in {-1, 1, 2} {
  \coordinate (tmp) at ($(o) + (0,\d)$);
  \draw (q1) -- (tmp);
  \draw (q2) -- (tmp);
  \fill[black] (tmp) circle (\graphdotsize);
}
\node[area,minimum height=4.3\oggs,minimum width=0.8\oggs,pin=266:$X_3$] at ($(o) + (0,0.5)$) {};  

\end{tikzpicture}

}
\subfigure[Examples of star graphs]{
  \begin{tikzpicture}[x=\graphgridsize,y=\graphgridsize] 

\coordinate (o) at (0,0);
\fill[black] (o) circle (\graphdotsize);
\foreach \d in {0, 90, ..., 270} {
  \coordinate (tmp) at ($(o) + (\d:1.5)$);
  \draw (o) -- (tmp);
  \fill[black] (tmp) circle (\graphdotsize);
}

\coordinate[right=4 of o] (o);
\coordinate (o1) at ($(o) + (0:0.3)$);
\coordinate (o2) at ($(o) + (180:0.3)$);
\fill[black] (o1) circle (\graphdotsize);
\fill[black] (o2) circle (\graphdotsize);
\draw (o1) -- (o2);
\foreach \d in {45, 135, 270} {
  \coordinate (tmp\d) at ($(o) + (\d:1.5)$);
  \draw (o2) -- (tmp\d);
  \fill[black] (tmp\d) circle (\graphdotsize);
}
\draw (o1) -- (tmp45);


\coordinate[right=4 of o] (o);
\coordinate (o1) at ($(o) + (0:0.3)$);
\coordinate (o2) at ($(o) + (180:0.3)$);
\fill[black] (o1) circle (\graphdotsize);
\fill[black] (o2) circle (\graphdotsize);
\draw (o1) -- (o2);
\foreach \d in {45, 135, 270} {
  \coordinate (tmp\d) at ($(o) + (\d:1.5)$);
  \draw (o2) -- (tmp\d);
  \fill[black] (tmp\d) circle (\graphdotsize);
}
\draw (o1) -- (tmp45);
\draw (o1) -- (tmp135);


\coordinate[right=4 of o] (o);
\coordinate (o1) at ($(o) + (0:0.3)$);
\coordinate (o2) at ($(o) + (180:0.3)$);
\fill[black] (o1) circle (\graphdotsize);
\fill[black] (o2) circle (\graphdotsize);
\draw (o1) -- (o2);
\foreach \d in {45, 135, 270} {
  \coordinate (tmp\d) at ($(o) + (\d:1.5)$);
  \draw (o1) -- (tmp\d);
  \draw (o2) -- (tmp\d);
  \fill[black] (tmp\d) circle (\graphdotsize);
}


\coordinate[right=3.5 of o] (o);
\coordinate (o1) at ($(o) + (90:0.3)$);
\coordinate (o2) at ($(o) + (210:0.5)$);
\coordinate (o3) at ($(o) + (330:0.5)$);
\fill[black] (o1) circle (\graphdotsize);
\fill[black] (o2) circle (\graphdotsize);
\fill[black] (o3) circle (\graphdotsize);
\draw (o1) -- (o2);
\draw (o1) -- (o3);
\draw (o2) -- (o3);
\foreach \d in {90, 270} {
  \coordinate (tmp\d) at ($(o) + (\d:1.5)$);
  \draw (o2) -- (tmp\d);
  \draw (o3) -- (tmp\d);
  \fill[black] (tmp\d) circle (\graphdotsize);
}
\draw (o1) -- (tmp90);


\coordinate[right=3 of o] (o);
\coordinate (o1) at ($(o) + (90:0.3)$);
\coordinate (o2) at ($(o) + (210:0.5)$);
\coordinate (o3) at ($(o) + (330:0.5)$);
\fill[black] (o1) circle (\graphdotsize);
\fill[black] (o2) circle (\graphdotsize);
\fill[black] (o3) circle (\graphdotsize);
\draw (o1) -- (o2);
\draw (o1) -- (o3);
\draw (o2) -- (o3);
\foreach \d in {90, 270} {
  \coordinate (tmp\d) at ($(o) + (\d:1.5)$);
  \draw (o2) -- (tmp\d);
  \draw (o3) -- (tmp\d);
  \fill[black] (tmp\d) circle (\graphdotsize);
}
\draw (o1) -- (tmp90);
\draw (o1) -- (tmp270);



\coordinate[right=3.5 of o] (o);
\coordinate (o1) at ($(o) + (18:1.3)$);
\coordinate (o2) at ($(o) + (90:1.3)$);
\coordinate (o3) at ($(o) + (162:1.3)$);
\coordinate (o4) at ($(o) + (234:1.3)$);
\coordinate (o5) at ($(o) + (306:1.3)$);
\foreach \d in {1, 2, ..., 5} {
  \fill[black] (o\d) circle (\graphdotsize);
}
\draw (o1) -- (o2);
\draw (o1) -- (o3);
\draw (o1) -- (o4);
\draw (o1) -- (o5);
\draw (o2) -- (o3);
\draw (o2) -- (o4);
\draw (o2) -- (o5);
\draw (o3) -- (o4);
\draw (o3) -- (o5);
\draw (o4) -- (o5);

\end{tikzpicture}

}
\caption{Examples of graphs draw with TikZ. These figures have been taken from a
course report for the graph theory course~\cite{FerryProblem}.}
\label{fig:tikz-examples}
\end{center}
\end{figure}

