\chapter{Conclusions}
\label{chapter:conslusions}

\section{Discussion}

WebSocket simplifies the development of applications with real-time capabilities as it provides a set of standardized methods for communication. This standardization is a huge improvement over former HTTP-based solution that tended to complicate the development of real-time applications as they are by nature workarounds to extend the functionality of HTTP. Another key benefit of WebSocket is that it enables communication as close as possible to real-time given the constrains of the web and physics (speed of light in fiber optic cables). This feature leads to a overall better user experience and network utilization when WebSocket matches the exact use case is was designed for.
\\ \\
To answer the first research question posed in Chapter~\ref{chapter:intro}, about the main challenges related to application development with WebSocket, one of the main takeaway from the work conducted here is that WebSocket substantially simplifies the development of applications with real-time functionality compared to previous approaches. The reason is that less workarounds are needed than in the case of HTTP-based communication which often require a considerable effort to generate a decent user experience. Application development with WebSocket can also lead to system architectures with less tightly coupled components and increased modularity as the standard supports higher-lever application protocols. Overall the standard is really well designed so there are no major challenges related to application development with WebSocket.
\\ \\
The second research question was directed towards the reliability of WebSocket in a "real world" production environment. As it stands, the main obstacle towards greater adoption of WebSocket is the current infrastructure of the Internet that is predominantly configured for HTTP traffic and tends to block any traffic that behaves differently. The WebSocket protocol was designed to be able to coexist with HTTP and deployed HTTP infrastructure but in multiple network environments that is not the case. There are solutions to the challenges that those network environments introduce for applications that utilize WebSocket which make it reliable enough for use in those "real world" production environments. The adoption of HTTP/2 in the near future will most likely benefit WebSocket as it requires networks to be configured to allow long-lived connections.
\\ \\
For the third research question, regarding the suitability of containerized virtualization to provide run-time environments for applications that utilize WebSocket, the experience has been fairly positive. The main WebSocket-related challenge when using containerized virtualization is that there are usually more network intermediaries, like proxies and load balancers, that must be specially configured for WebSocket traffic. Otherwise the challenges are same when deploying WebSocket applications in standard VMs. Among the opportunities WebSocket introduces for containerized infrastructure is the possibility to reduce network load as HTTP APIs seem to be the current dominant form of communication between containers in production systems built on the microservices principles \cite{fowlervMicroservices}. However, Using WebSocket for container communication would require a different mindset as the protocol is stateful opposed to HTTP which is stateless.
\\ \\
Summing up, WebSocket has a tremendous potential for become one of the core protocols of the web as it is designed for a use case that is constantly become more important. There are currently challenges related to the widespread adoption of the standard but the web seems to be heading into a direction that benefits WebSocket. These factors might enable WebSocket to live up to its great promise as the foundation for real-time communication on the web.

\section{Further Work}

The logical next stop of this thesis work would be to deploy the WebSocket messaging system in a production environment using containerized virtualization and monitor its performance and reliability. There is only so much that can be learned about a software system without exposing it to "real world" use and there are most certainly issues with the current implementation of the system that have not been identified yet. As mentioned before, one of the main advantages of microservices architecture is the possibility of technology heterogeneity which can be used for the adaptation of new technologies and implementations. For a system like the one presented is this thesis this can be done without affecting the rest of systems with which it interacts while they run in production. In addition to getting a better understanding of the WebSocket messaging system this would lead to a better understanding of operational challenges related to WebSocket-based infrastructure which will be crucial if the protocol achieves widespread adoption. 