\chapter{Conclusions}
\label{chapter:conslusions}

WebSocket simplifies the development of applications with real-time capabilities as it provides a set of standardized methods for communication. This standardization is a huge improvement over former HTTP-based solution that tended to complicate the development of real-time applications as they are by nature workarounds to extend the functionality of HTTP. Another key benefit of WebSocket is that it enables communication as close as possible to real-time given the constrains of the web and physics (speed of light in fiber optic cables). This feature leads to a overall better user experience and network utilization when WebSocket matches the exact use case is was designed for.
\\ \\
To answer the first research question posed in Chapter~\ref{chapter:intro}, about the main challenges related to application development with WebSocket, it has been the experience of this thesis work that WebSocket substantially simplifies the development of applications with real-time functionality compared to previous approaches. The reason is that less workarounds are needed than in the case of HTTP-based communication which often require a considerable effort to generate a decent user experience. Application development with WebSocket can also lead to system architectures with less tightly coupled components and increased modularity as the standard supports higher-lever application protocols. Overall the standard is really well designed so there are no major challenges related to application development with WebSocket.
\\ \\
The second research question was directed towards the reliability of WebSocket in a "real world" production environment. As it stands, the main obstacle towards greater adoption of WebSocket is the current infrastructure of the Internet that is predominantly configured for HTTP traffic and tends to block any traffic that behaves differently. The WebSocket protocol was designed to be able to coexist with HTTP and deployed HTTP infrastructure but in multiple network environments that is not the case. There are solutions to the challenges that those network environments introduce for applications that utilize WebSocket which make it reliable enough for use in those "real world" production environments. The adoption of HTTP/2 in the near future will most likely benefit WebSocket as it requires networks to be configured to allow long-lived connections.
\\ \\
Summing up, WebSocket has a tremendous potential for become one of the core protocols of the web as it is designed for a use case that is constantly become more important. There are currently challenges related to the widespread adoption of the standard but the web seems to be heading into a direction that benefits WebSocket. These factors might enable WebSocket to live up to its great promise as the foundation for real-time communication on the web.