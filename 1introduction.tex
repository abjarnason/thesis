\chapter{Introduction}
\label{chapter:intro}

The value of communication mediums tends to be determined how fast they can transfer information and connect two or more correspondents. We have come to expect real-time interaction, for voice and text communication, over the Internet as a regular thing and its importance will undoubtedly  grow in the not so distant future. As the Internet becomes the predominant communication channel for voice and text interaction, at the expense of more traditional telecommunication systems, the underlying technology of the web will have to be reviewed and possibly extended. This is where the WebSocket standard comes into the picture as a possible foundation for real-time data transfer on the web. Whether that turns out to be the case or not, there are sure exiting times ahead as the Internet becomes the predominant form of communication for the world.

\section{Thesis Context}

This thesis is carried out at IDT Messaging, a technology company that develops and maintains a platform for messaging apps. The work conducted here is seen as a starting point for the further improvement of a chat client for web, developed and operated by IDT Messaging. The thesis is a part of the EIT ICT Labs master program in Distributed Systems and Services.

\section{Research Scope and Goals}

As en emerging technology, WebSocket has the potential to make communication on the web more effective and efficient. As such, WebSocket offers great promises but since this technology is still relatively young and not quite proven in production environments it has yet to establish itself as the definitive standard for real-time communication on the web. The goal of this thesis is to get more familiar with WebSocket through the design, implementation and deployment of a WebSocket messaging system intended for chat applications and use the acquired experience to answer the following research questions

\begin{itemize}
\item What are the main challenges related to application development with WebSocket?
\item How reliable is WebSocket in "real world" production environments?
\end{itemize}

\section{Thesis Structure}

Chapter 2 begins with a historical background of data transfer on the web and the limitations of current approaches for real-time communication on the web. The WebSocket standard is then introduced as a possible refinement for event-driven communication on the web and a standard for the real-time web. Finally, alternative technology standards are presented that have the potential to challenge WebSocket for specific types of real-time communication on the web.
\\ \\
Chapter 3 provides a general introduction to microservices and pinpoints potential benefits and challenges. The chapter also covers asynchronous communication patterns and containerized virtualization which are concepts used in the implementation of a messaging system presented in the following chapter.
\\ \\
In Chapter 4 a messaging system is introduced that enables clients to communicate with a back end service with the WebSocket protocol. The tools used to build the system are introduced along with design and implementation decision. Afterwards, results of benchmarking tests are presented which are seen as the foundation for further development along with ideas for further improvement of the messaging system.
\\ \\
Chapter 5 discusses challenges related to the deployment of WebSocket applications and considerations for containerized infrastructure from the perspective of the WebSocket protocol. These considerations are based on empirical knowledge acquired during the development of the messaging system from Chapter 4 and related experiments.
\\ \\
Chapter 6 concludes the thesis and presents ideas for further work and discusses possible next steps related to the work carried out as a part of this thesis.